\section{Teamwork}

 \subsection{Group meetings}
 
 In order to work together as a group we have used several well-known processes and tools.\newline
 
We have had group meetings, normally once a week but some times meetings twice a week were necessarily. This happened during the first stages of the project when we were planning the work and formalizing the requirements. The first meetings were fairly long, as there were numerous ideas and contributions by all team members. However, these meetings were successful as most of the times we managed to agreed on each other?s views and hence produced realistic short-term plans and task. \newline
 
The meetings were informal, but there was always an agenda to be followed. A team member was always responsible for taking the minutes of the meetings and uploading them on the online group work repository, such as Dropbox. The agendas created were crucial to guide the meetings and the project. The priority of these, discussed the progress of the team and well as individual work done by members. Based on the work and reports delivered, we proceeded to address problems arising and give attention to areas where more work was required. We concluded the meetings planning on tasks for the following week and points to be discussed for next time, including time and date.

 \subsection{Requirements and Design processes}
 The Requirements and Design processes were completed fairly successful, since we worked and communicate frequently and also provided feedback, which was further improved.\newline
 
During the first stages of the system lifecycle, the requirements were vague and hence there were different interpretations by team members. Nonetheless, we analyzed the systems specifications and different traffic management policies and reached an agreement to produce two types requirements, as explained in section two. Consequently, we decided to go ahead developing with the primary requirements. Though, we also took into consideration that we could have to adjust the requirements and design, according to progress of implementation process and the whole project itself. \newline
 
For this process we worked the five of us together, sometimes the worked was produced in the computer laboratories or in the library, where we booked the study group rooms. We have used various types of tools and technologies, such as UML, Microsoft Project to produce diagrams and charts. Much simpler techniques, such as pen and paper were used to produce uses-cases and to model the classes, objects and the entire systems. In several occasions, we were required to do individual research and produce requirements based on it and share the work on Dropbox.

\subsection{Implementation and testing} 
For these process the work was produced slightly different. Since work was divided into small groups, each team had its own way to approach respective task. \newline

The team of programmers has been implementing the requirements, as a team of two people by meeting up several times a week. They have also been producing worked individually at their own time. Most of the worked was completed at the KCL labs facilities as well as using members own computers and properties. In order to communicate outside university hours, the two members have used social media channels, such as Skype and whatsapp. The programmers have coordinated fairly well, as they have used GitHub, in a professional manner to share their code and to make any changes as needed. \newline

Similarly, the team of Testers has been also meeting in the library at group study rooms. Sections of half a day have been spent and a substantive effort was put on testing the different components of the system. In addition, testers were required to do some preparation individually by studying certain parts of the code, before meeting up, so progress could be made during the testing meetings. During the Testing sessions, specific test cases were tested for finding the bugs on certain functions and classes. The group also used Skype and whatsapp for communication purposes and to organize the work. 

\subsection{Documentation and main tools for collaboration} 

For the documentation process, work has been produced individually, however, everyone has contributed equally and also helped each other by providing useful feedback. Certain parts of this final report were produced according to members' responsibilities with the project. For example, programmers have written the implementation part, as they have full understanding of the code and analysts have conducted the literature review because this task was fairly independent from other processes of the project or system. In addition, short meetings were scheduled once a week to discuss the progress of the final report and documentation. Short-term deadlines were set for this process, in order to encourage everyone to produce a good quality work and not leave the majority of work for the last minute.\newline

Upon completion of the tasks, members uploaded the work on Dropbox to the group's repository and one person was in charge of putting all the parts together, and formatting the report using Latex mark-up language.\newline  

For this process, the communications channels mention above have been used significantly to coordinate with the different groups as well as all team members. The following  explains the main tools utilised to complete the project: 

\begin{enumerate}
\item GitHub
\begin{enumerate}
\item Have been used to share the source code and update the progression of the software implementation.

\end{enumerate} 
\item Dropbox
\begin{enumerate}
\item Have been used to record meeting minutes, share references and documentation, update each member of group of each meeting in cases where they miss the meeting.

\end{enumerate} 
\item Skype 
\begin{enumerate}
\item Have been used for conference calls, and in order to consume time for meeting arrangements.

\end{enumerate} 
\item Whatsapp 
\begin{enumerate}
\item Have been used to inform the group when we had meetings and for instant chatting and updates.

\end{enumerate} 
\item Latex 
\begin{enumerate}
\item Have been used to put all the work and documentation together and formatting the report in a well presentable way. 
 
\end{enumerate} 
\end{enumerate}
  
  
